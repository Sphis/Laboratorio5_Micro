\section{Conclusiones y recomendaciones}
Las conclusiones para este trabajo son:
\begin{itemize}
\item Los ejemplos brindados fueron de gran ayuda para realizar las pruebas con el giroscopio.
\item El notebook de Colab está muy bien hecho porque es rápido, lo cual tiene sentido al tratarse de Python, y realiza un muy buen entrenamiento de los datos porque a la hora de llevar el archivo \texttt{model.h} al código \texttt{IMU\_ Classifier}, resulta que que la placa es consistente cuando asigna una probabilidad al movimiento realizado previamente.
\item Tensorflow lite es una herramienta muy poderosa porque permite hacer uso de técnicas de machine learning en dispositivos que no tienen un rendimiento tan alto.
\item Por tanto, el programa cumple satisfactoriamente porque los movimientos realizados fueron hechos con la misma cantidad de muestras; 1750. Para el caso del puñetazo se hizo en una misma dirección hacia la pantalla de la PC. En cambio, los otros dos: movimiento circular se ejecutó en sentido horario y anti-horario, y el brazo hacia arriba se hizo en diferentes ejes. Además, el buen entrenamiento de los datos mencionado previamente ayudó mucho para identificar los movimientos realizados.
\end{itemize}
A modo de recomendación, prestar mucha atención en clases con \textbf{todo} lo que menciona el profesor, por supuesto estudiar muy bien los ejemplos recomendados por él así como la literatura. Lo otro muy importante es hacer los movimientos con diferentes ejes y así no tener un mismo patrón en un muestreo de 1750 muestras ya que no habrá diversidad para identificar el movimiento realizado.
